% Options for packages loaded elsewhere
\PassOptionsToPackage{unicode}{hyperref}
\PassOptionsToPackage{hyphens}{url}
\PassOptionsToPackage{dvipsnames,svgnames,x11names}{xcolor}
%
\documentclass[
  letterpaper,
  DIV=11,
  numbers=noendperiod]{scrartcl}

\usepackage{amsmath,amssymb}
\usepackage{iftex}
\ifPDFTeX
  \usepackage[T1]{fontenc}
  \usepackage[utf8]{inputenc}
  \usepackage{textcomp} % provide euro and other symbols
\else % if luatex or xetex
  \usepackage{unicode-math}
  \defaultfontfeatures{Scale=MatchLowercase}
  \defaultfontfeatures[\rmfamily]{Ligatures=TeX,Scale=1}
\fi
\usepackage{lmodern}
\ifPDFTeX\else  
    % xetex/luatex font selection
\fi
% Use upquote if available, for straight quotes in verbatim environments
\IfFileExists{upquote.sty}{\usepackage{upquote}}{}
\IfFileExists{microtype.sty}{% use microtype if available
  \usepackage[]{microtype}
  \UseMicrotypeSet[protrusion]{basicmath} % disable protrusion for tt fonts
}{}
\makeatletter
\@ifundefined{KOMAClassName}{% if non-KOMA class
  \IfFileExists{parskip.sty}{%
    \usepackage{parskip}
  }{% else
    \setlength{\parindent}{0pt}
    \setlength{\parskip}{6pt plus 2pt minus 1pt}}
}{% if KOMA class
  \KOMAoptions{parskip=half}}
\makeatother
\usepackage{xcolor}
\setlength{\emergencystretch}{3em} % prevent overfull lines
\setcounter{secnumdepth}{3}
% Make \paragraph and \subparagraph free-standing
\ifx\paragraph\undefined\else
  \let\oldparagraph\paragraph
  \renewcommand{\paragraph}[1]{\oldparagraph{#1}\mbox{}}
\fi
\ifx\subparagraph\undefined\else
  \let\oldsubparagraph\subparagraph
  \renewcommand{\subparagraph}[1]{\oldsubparagraph{#1}\mbox{}}
\fi


\providecommand{\tightlist}{%
  \setlength{\itemsep}{0pt}\setlength{\parskip}{0pt}}\usepackage{longtable,booktabs,array}
\usepackage{calc} % for calculating minipage widths
% Correct order of tables after \paragraph or \subparagraph
\usepackage{etoolbox}
\makeatletter
\patchcmd\longtable{\par}{\if@noskipsec\mbox{}\fi\par}{}{}
\makeatother
% Allow footnotes in longtable head/foot
\IfFileExists{footnotehyper.sty}{\usepackage{footnotehyper}}{\usepackage{footnote}}
\makesavenoteenv{longtable}
\usepackage{graphicx}
\makeatletter
\def\maxwidth{\ifdim\Gin@nat@width>\linewidth\linewidth\else\Gin@nat@width\fi}
\def\maxheight{\ifdim\Gin@nat@height>\textheight\textheight\else\Gin@nat@height\fi}
\makeatother
% Scale images if necessary, so that they will not overflow the page
% margins by default, and it is still possible to overwrite the defaults
% using explicit options in \includegraphics[width, height, ...]{}
\setkeys{Gin}{width=\maxwidth,height=\maxheight,keepaspectratio}
% Set default figure placement to htbp
\makeatletter
\def\fps@figure{htbp}
\makeatother

% load packages
\usepackage{geometry}
\usepackage{xcolor}
\usepackage{eso-pic}
\usepackage{fancyhdr}
\usepackage{sectsty}
\usepackage{fontspec}
\usepackage{titlesec}

%% Set page size with a wider right margin
\geometry{a4paper, total={170mm,257mm}, left=20mm, top=20mm, bottom=20mm, right=50mm}

%% Let's define some colours
\definecolor{light}{HTML}{E6E6FA}
\definecolor{highlight}{HTML}{800080}
\definecolor{dark}{HTML}{330033}

%% Let's add the border on the right hand side 
\AddToShipoutPicture{% 
    \AtPageLowerLeft{% 
        \put(\LenToUnit{\dimexpr\paperwidth-3cm},0){% 
            \color{light}\rule{3cm}{\LenToUnit\paperheight}%
          }%
     }%
     % logo
    %\AtPageLowerLeft{% start the bar at the bottom right of the page
    %    \put(\LenToUnit{\dimexpr\paperwidth-2.25cm},27.2cm){% move it to the top right
    %        \color{light}\includegraphics[width=1.5cm]{_extensions/nrennie/PrettyPDF/logo.png}
    %      }%
    % }%
}

%% Style the page number
\fancypagestyle{mystyle}{
  \fancyhf{}
  \renewcommand\headrulewidth{0pt}
  \fancyfoot[R]{\thepage}
  \fancyfootoffset{3.5cm}
}
\setlength{\footskip}{20pt}

%% style the chapter/section fonts
\chapterfont{\color{dark}\fontsize{20}{16.8}\selectfont}
\sectionfont{\color{dark}\fontsize{20}{16.8}\selectfont}
\subsectionfont{\color{dark}\fontsize{14}{16.8}\selectfont}
\titleformat{\subsection}
  {\sffamily\Large\bfseries}{\thesection}{1em}{}[{\titlerule[0.8pt]}]
  
% left align title
\makeatletter
\renewcommand{\maketitle}{\bgroup\setlength{\parindent}{0pt}
\begin{flushleft}
  {\sffamily\huge\textbf{\MakeUppercase{\@title}}} \vspace{0.3cm} \newline
  {\Large {\@subtitle}} \newline
  \@author
\end{flushleft}\egroup
}
\makeatother

%% Use some custom fonts
%%\setsansfont{Ubuntu}[
%%    Path=_extensions/nrennie/PrettyPDF/Ubuntu/,
%%    Scale=0.9,
%%    Extension = .ttf,
%%    UprightFont=*-Regular,
%%    BoldFont=*-Bold,
%%    ItalicFont=*-Italic,
%%    ]
%%
%%\setmainfont{Ubuntu}[
%%    Path=_extensions/nrennie/PrettyPDF/Ubuntu/,
%%    Scale=0.9,
%%    Extension = .ttf,
%%    UprightFont=*-Regular,
%%    BoldFont=*-Bold,
%%    ItalicFont=*-Italic,
%%    ]
\KOMAoption{captions}{tableheading}
\makeatletter
\makeatother
\makeatletter
\makeatother
\makeatletter
\@ifpackageloaded{caption}{}{\usepackage{caption}}
\AtBeginDocument{%
\ifdefined\contentsname
  \renewcommand*\contentsname{Table of contents}
\else
  \newcommand\contentsname{Table of contents}
\fi
\ifdefined\listfigurename
  \renewcommand*\listfigurename{List of Figures}
\else
  \newcommand\listfigurename{List of Figures}
\fi
\ifdefined\listtablename
  \renewcommand*\listtablename{List of Tables}
\else
  \newcommand\listtablename{List of Tables}
\fi
\ifdefined\figurename
  \renewcommand*\figurename{Figure}
\else
  \newcommand\figurename{Figure}
\fi
\ifdefined\tablename
  \renewcommand*\tablename{Table}
\else
  \newcommand\tablename{Table}
\fi
}
\@ifpackageloaded{float}{}{\usepackage{float}}
\floatstyle{ruled}
\@ifundefined{c@chapter}{\newfloat{codelisting}{h}{lop}}{\newfloat{codelisting}{h}{lop}[chapter]}
\floatname{codelisting}{Listing}
\newcommand*\listoflistings{\listof{codelisting}{List of Listings}}
\makeatother
\makeatletter
\@ifpackageloaded{caption}{}{\usepackage{caption}}
\@ifpackageloaded{subcaption}{}{\usepackage{subcaption}}
\makeatother
\makeatletter
\@ifpackageloaded{tcolorbox}{}{\usepackage[skins,breakable]{tcolorbox}}
\makeatother
\makeatletter
\@ifundefined{shadecolor}{\definecolor{shadecolor}{rgb}{.97, .97, .97}}
\makeatother
\makeatletter
\@ifundefined{codebgcolor}{\definecolor{codebgcolor}{named}{light}}
\makeatother
\makeatletter
\makeatother
\ifLuaTeX
  \usepackage{selnolig}  % disable illegal ligatures
\fi
\IfFileExists{bookmark.sty}{\usepackage{bookmark}}{\usepackage{hyperref}}
\IfFileExists{xurl.sty}{\usepackage{xurl}}{} % add URL line breaks if available
\urlstyle{same} % disable monospaced font for URLs
\hypersetup{
  pdftitle={ESRM 6553: Adv. Multivariate},
  pdfauthor={Dr.~Jihong Zhang},
  colorlinks=true,
  linkcolor={highlight},
  filecolor={Maroon},
  citecolor={Blue},
  urlcolor={highlight},
  pdfcreator={LaTeX via pandoc}}

\title{ESRM 6553: Adv. Multivariate}
\usepackage{etoolbox}
\makeatletter
\providecommand{\subtitle}[1]{% add subtitle to \maketitle
  \apptocmd{\@title}{\par {\large #1 \par}}{}{}
}
\makeatother
\subtitle{Spring 2024, Mondays, 5:00-7:45PM, Classroom GRAD 0229}
\author{Dr.~Jihong Zhang}
\date{2024-01-13}

\begin{document}
\maketitle
\pagestyle{mystyle}

\ifdefined\Shaded\renewenvironment{Shaded}{\begin{tcolorbox}[boxrule=0pt, borderline west={3pt}{0pt}{shadecolor}, frame hidden, colback={codebgcolor}, enhanced, breakable, sharp corners]}{\end{tcolorbox}}\fi

\hypertarget{general-information}{%
\section{General Information}\label{general-information}}

\begin{itemize}
\tightlist
\item
  \textbf{Course Code:} ESRM 6553
\item
  \textbf{Course ID:} 026376
\item
  \textbf{Course time and location:} Mon 17:00-19:45; GRAD 229
\item
  \textbf{Instructor:} Jihong Zhang
\item
  \textbf{Contact Information:} jzhang@uark.edu
\item
  \textbf{Office Location:} GRAD 0109
\item
  \textbf{Office Hours:} Tu 1:30-4:30PM
\item
  \textbf{Office Phone} +1 479-575-5235
\item
  \textbf{Classroom:} GRAD 229
\item
  \textbf{Semester:} Spring 2024
\item
  \textbf{Credits:} 3 credit hours
\end{itemize}

\hypertarget{course-topics}{%
\subsection{Course Topics}\label{course-topics}}

\begin{enumerate}
\def\labelenumi{\arabic{enumi}.}
\tightlist
\item
  \textbf{Introduction to Bayesian Statistics}

  \begin{itemize}
  \tightlist
  \item
    Likelihoods;
  \item
    Probability distributions, such as priors and posteriors;
  \item
    Theorectial foundations of Bayesian infernce;
  \end{itemize}
\item
  \textbf{Bayesian Inference and Computational Methods}

  \begin{itemize}
  \tightlist
  \item
    Markov Chain Monte Carlo (MCMC) estimation procedures;
  \end{itemize}
\item
  \textbf{Bayesian Modeling Evaluation}

  \begin{itemize}
  \tightlist
  \item
    Model specification, estimation, and testing.
  \item
    Model fit and model comparison, such as convergence diagnostics,
    posterior predictive checks, information criteria
  \end{itemize}
\item
  \textbf{Bayesian Modeling}

  \begin{itemize}
  \tightlist
  \item
    Estimating and making inferences from psychometric models such as
    Confirmatory Factor Analysis (CFA) or Item Response Theory (IRT)
    models
  \end{itemize}
\item
  \textbf{Advanced Topics in Bayesian Multivariate Analysis}

  \begin{itemize}
  \tightlist
  \item
    Bayesian networks.
  \item
    Multilevel/Hierarchical models, mixture models.
  \item
    Missing data, non-normal data.
  \end{itemize}
\end{enumerate}

\hypertarget{course-description}{%
\subsection{Course Description}\label{course-description}}

This course offers an in-depth exploration of multivariate statistics
within the context of Bayesian inferences. Bayesian statistics have been
widely used in public health, education, and psychology. Bayesian
techniques are increasingly used in Artificial Intelligence and Brain
models for decision-making under uncertainty. Designed for graduate
students in educational statistics and research methods, it focuses on
the theoretical underpinnings and practical applications of Bayesian
approaches in psychometric modeling. Prerequisites include basic
knowledge of multivariate statistics and psychometrics.

\hypertarget{course-objectives}{%
\subsection{Course Objectives}\label{course-objectives}}

Upon completion of ESRM 6554 - Adv. Multivariate, students will:

\begin{enumerate}
\def\labelenumi{\arabic{enumi}.}
\tightlist
\item
  Comprehend fundamental concepts and principles of Bayesian
  multivariate analysis;
\item
  Articulate the rationale of Bayesian approaches to data analysis and
  statistical inference;
\item
  Compare Bayesian inference to MLE;
\item
  Develop conceptual and mathematical Bayesian literacy, as well as
  computer software skills (e.g., R, Stan, or JAGS) required to conduct
  Bayesian data analyses in educational research;
\item
  Gain technical foundations necessary to be contributors to applied and
  methodological research that use Bayesian methods;
\item
  Conduct analyses on empirical data, interpret results, and communicate
  work in written and oral presentations.
\end{enumerate}

\hypertarget{prerequisite-knowledge}{%
\subsection{Prerequisite Knowledge}\label{prerequisite-knowledge}}

It is assumed that students have has solid statistical training up to
and including topics in multivariate statistics (ESRM 6413, 6423, and
6453). In addition, it is assumed you are familiar with R programming
(python or SAS are fine). SPSS may not be sufficient for this course.

\begin{itemize}
\tightlist
\item
  Lectures for theoretical understanding.
\item
  Hands-on sessions with statistical software.
\item
  Group discussions and presentations.
\item
  Research project guidance.
\end{itemize}

\hypertarget{how-to-be-successful-in-this-class}{%
\subsection{How to Be Successful in This
Class}\label{how-to-be-successful-in-this-class}}

\begin{itemize}
\tightlist
\item
  Come to class ready to learn.
\item
  Complete the out-of-class exercises prior to each class.
\item
  If you become confused or don't fully grasp a concept, ask for help
  from your instructor
\item
  Know what is going on: keep up with email, course announcements, and
  the course schedule.
\item
  Try to apply the information to your area of interest --- if you have
  a cool research idea, come talk to me!
\end{itemize}

\hypertarget{course-materials}{%
\section{Course Materials}\label{course-materials}}

\hypertarget{required-materials}{%
\subsection{Required Materials}\label{required-materials}}

\begin{itemize}
\tightlist
\item
  \textbf{Primary Text:} Richard McElreath (2019),
  \href{https://github.com/Booleans/statistical-rethinking/blob/master/Statistical\%20Rethinking\%202nd\%20Edition.pdf}{\emph{Statistical
  Rethinking: A Bayesian Course with Examples in R and Stan}}. Free to
  download it online.
\item
  \textbf{Primary Text:} Levy, Mislevy (2016),
  \href{https://www.routledge.com/Bayesian-Psychometric-Modeling/Levy-Mislevy/p/book/9780367737092}{\emph{Bayesian
  Psychometric Modeling}}. Chapters will be uploaded before class.
\end{itemize}

\hypertarget{optional-materials}{%
\subsection{Optional Materials}\label{optional-materials}}

\begin{itemize}
\tightlist
\item
  Kaplan, D. (2014), Bayesian Statistics for the Social Sciences. New
  York: Guilford Press.
\item
  Gelman, A., Carlin, J. B., Stern, H. S., and Rubin, D. B. (2020),
  Bayesian Data Analysis 3rd edition. Chapman and Hall.
\item
  \href{http://andrewgelman.com}{Andrew Gelman's Website} for an
  unfiltered, stream of consciousness Bayesian commentary
\item
  \textbf{Supplementary Texts:}
  \href{https://mc-stan.org/docs/2_33/stan-users-guide-2_33.pdf}{Stan
  User's Guide}
\end{itemize}

\hypertarget{software}{%
\subsection{Software}\label{software}}

\begin{itemize}
\tightlist
\item
  R and R packages (tidyverse)
\item
  Stan is gaining in popularity and has an avid user community. To use
  Stan in R, you need to download RStan package. Here is the
  \href{https://github.com/stan-dev/rstan/wiki/RStan-Getting-Started}{tutorial}
  of installing RStan.
\item
  (Optional) Mplus, JAGS,
  \href{http://www.mrc-bsu.cam.ac.uk/software/bugs/the-bugs-project-winbugs/}{WinBUGS}
\end{itemize}

\hypertarget{assignments}{%
\section{Assignments}\label{assignments}}

\hypertarget{projects}{%
\subsection{Projects}\label{projects}}

Students will complete a project utilizing your knowledge learnt from
the class. You may work individually. I will provide data and questions
for this project OR you can use data that is of interest to you in your
GA position or dissertation research. The primary objective of the
research project is to facilitate the application and understanding of
concepts learned in this course.

There will be a short project proposal due around week 12 - it can be
sooner if you want to get started early. Required for this proposal is
an NCME-type conference proposal (800 words maximum). Please see
\href{https://higherlogicdownload.s3.amazonaws.com/NCME/4b7590fc-3903-444d-b89d-c45b7fa3da3f/UploadedImages/2024_Annual_Meeting/NCME_2024_Call_for_Proposals_Final.pdf}{Individual
Paper Presentations} for more details

Typical components of the research proposal include:

\begin{itemize}
\tightlist
\item
  Title (no more than 12 words)
\item
  Summary of research (no more than 800 words)

  \begin{itemize}
  \tightlist
  \item
    Background of research
  \item
    Research questions/hypotheses
  \item
    Method (Data, Analysis Plan)
  \item
    Preliminary findings
  \item
    References/Table/Figure
  \end{itemize}
\end{itemize}

\hypertarget{brief-quiz}{%
\subsection{Brief quiz}\label{brief-quiz}}

At the commencement of each class session, a brief quiz consisting of
one to three questions will be administered. These quizzes are intended
for formative assessment purposes only and will not contribute to your
overall score. However, regular attendance is essential, as it will
ensure full credit in the final grading process.

\hypertarget{grading}{%
\subsection{Grading}\label{grading}}

\begin{enumerate}
\def\labelenumi{\arabic{enumi}.}
\tightlist
\item
  Brief quiz: 60\%
\item
  Project Presentation: 20\%
\item
  Project Proposal: 20\%
\end{enumerate}

\hypertarget{academic-policies}{%
\section{Academic Policies}\label{academic-policies}}

\hypertarget{ai-statement}{%
\subsection{AI Statement}\label{ai-statement}}

Specific permissions will be provided to students regarding the use of
generative artificial intelligence tools on certain graded activities in
this course. In these instances, I will communicate explicit permission
as well as expectations and any pertinent limitations for use and
attribution. Without this permission, the use of generative artificial
intelligence tools in any capacity while completing academic work
submitted for credit, independently or collaboratively, will be
considered academic dishonesty and reported to the Office of Academic
Initiatives and Integrity.

\hypertarget{academic-integrity}{%
\subsection{Academic Integrity}\label{academic-integrity}}

You are responsible for reading and understanding the University of
Arkansas' Academic Integrity Policy. You are expected to complete all
assignments and exams with the highest level of integrity. Any form of
academic dishonesty will result in a failing grade for the course and
will be reported to the Office of Academic Integrity. If you have any
questions about what constitutes academic dishonesty, please ask me.

\hypertarget{emergency-preparadness}{%
\subsection{Emergency Preparadness}\label{emergency-preparadness}}

The University of Arkansas is committed to providing a safe and healthy
environment for study and work. In that regard, the university has
developed a campus safety plan and an emergency preparedness plan to
respond to a variety of emergency situations. The emergency preparedness
plan can be found at emergency.uark.edu. Additionally, the university
uses a campus-wide emergency notification system, UARKAlert, to
communicate important emergency information via email and text
messaging. To learn more and to sign up:
http://safety.uark.edu/emergency-preparedness/emergency-notification-system/

\hypertarget{inclement-weather}{%
\subsection{Inclement Weather}\label{inclement-weather}}

Each faculty member is responsible for determining whether or not to
cancel class due to inclement weather. If you have any questions about
whether or not class will be held, please contact me. If I cancel class,
I will notify you via email and/or Blackboard. In general, students need
to know how and when they will be notified in the event that class is
cancelled for weather-related reasons. Please see
\href{http://safety.uark.edu/inclement-weather/}{here} for more
information.

\hypertarget{academic-support}{%
\subsection{Academic Support}\label{academic-support}}

A complete list and brief description of academic support programs can
be found on the University's Academic Support site, along with links to
the specific services, hours, and locations. Faculty are encouraged to
be familiar with these programs and to assist students with finding an
using the support services that will help them be successful. Please see
\href{http://www.uark.edu/academics/academic-support.php}{here} for more
information.

\hypertarget{religious-holidays}{%
\subsection{Religious Holidays}\label{religious-holidays}}

The university does not observe religious holidays; however Campus
Council has passed the following resolution concerning individual
observance of religious holidays and class attendance:

\begin{quote}
When members of any religion seek to be excused from class for religious
reasons, they are expected to provide their instructors with a schedule
of religious holidays that they intend to observe, in writing, before
the completion of the first week of classes.
\end{quote}

\hypertarget{schedule}{%
\subsection{Schedule}\label{schedule}}

Weekly breakdown of topics and readings:

\begin{longtable}[]{@{}
  >{\centering\arraybackslash}p{(\columnwidth - 6\tabcolsep) * \real{0.0577}}
  >{\centering\arraybackslash}p{(\columnwidth - 6\tabcolsep) * \real{0.0673}}
  >{\raggedright\arraybackslash}p{(\columnwidth - 6\tabcolsep) * \real{0.5192}}
  >{\raggedright\arraybackslash}p{(\columnwidth - 6\tabcolsep) * \real{0.3558}}@{}}
\toprule\noalign{}
\begin{minipage}[b]{\linewidth}\centering
Week
\end{minipage} & \begin{minipage}[b]{\linewidth}\centering
Date
\end{minipage} & \begin{minipage}[b]{\linewidth}\raggedright
Topic
\end{minipage} & \begin{minipage}[b]{\linewidth}\raggedright
Reading
\end{minipage} \\
\midrule\noalign{}
\endhead
\bottomrule\noalign{}
\endlastfoot
1 & 01/15 & No Class & \\
2 & 01/22 & Introduction to Bayesian Statistics Part I & SR\footnote{SR:
  \emph{Statistical Rethinking 2ed Edition} by Richard McElreath}
Chapter 1 \\
3 & 01/29 & Introduction to Bayesian Statistics Part II & SR Chapter
2 \\
4 & 02/05 & Bayesian Inference and Computational Methods I & SR Chapter
3 \\
5 & 02/12 & Bayesian Inference and Computational Methods II & SR Chapter
3 \\
6 & 02/19 & Bayesian Inference and Computational Methods III & SR
Chapter 9.2 \& 9.3 \\
7 & 02/26 & Bayesian Modeling Evaluation I: Model diagnosis & SR Chapter
7.1 \& BPM\footnote{BPM: \emph{Bayesian Psychometric Modeling} by Levy,
  Mislevy (2016)} Chapter 10 \\
8 & 03/04 & Bayesian Modeling Evaluation II & SR Chapter 7.4 \& 7.5 \\
9 & 03/11 & Bayesian Modeling I: CFA & BPM Chapter 9 \\
10 & 03/18 & Spring Break & \\
11 & 03/25 & Bayesian Modeling II: IRT & BPM Chapter 11 \\
12 & 04/01 & Bayesian Modeling III: CDM & BPM Chapter 13 \\
13 & 04/08 & Advanced Topics I: Non-normal data & \\
14 & 04/15 & Advanced Topics II: Multilevel & SR Chapter 11 \\
15 & 04/22 & Advanced Topics III: Missing Data, Measurement Error & SR
Chapter 15 \\
16 & 04/29 & Student Presentations & \\
\end{longtable}

Academic calendar for Spring 2024:
\href{https://registrar.uark.edu/academic-dates/academic-semester-calendar/spring-2024-january-intersession-2024.php}{Here}



\end{document}
